%%%%%%%%%%%%%%%%%%%%%%%%%%%%%%%%%%%%%%%%%
% Short Sectioned Assignment
% LaTeX Template
% Version 1.0 (5/5/12)
%
% This template has been downloaded from:
% http://www.LaTeXTemplates.com
%
% Original author:
% Frits Wenneker (http://www.howtotex.com)
%
% License:
% CC BY-NC-SA 3.0 (http://creativecommons.org/licenses/by-nc-sa/3.0/)
%
%%%%%%%%%%%%%%%%%%%%%%%%%%%%%%%%%%%%%%%%%

%----------------------------------------------------------------------------------------
%	PACKAGES AND OTHER DOCUMENT CONFIGURATIONS
%----------------------------------------------------------------------------------------

\documentclass[paper=a4, fontsize=11pt]{scrartcl} % A4 paper and 11pt font size

\usepackage[T1]{fontenc} % Use 8-bit encoding that has 256 glyphs
%\usepackage{fourier} % Use the Adobe Utopia font for the document - comment this line to return to the LaTeX default
\usepackage[english]{babel} % English language/hyphenation
\usepackage{amsmath,amsfonts,amsthm} % Math packages

\usepackage{lipsum} % Used for inserting dummy 'Lorem ipsum' text into the template

\usepackage{sectsty} % Allows customizing section commands
\allsectionsfont{\centering \normalfont\scshape} % Make all sections centered, the default font and small caps

\usepackage{fancyhdr} % Custom headers and footers
\pagestyle{fancyplain} % Makes all pages in the document conform to the custom headers and footers
\fancyhead{} % No page header - if you want one, create it in the same way as the footers below
\fancyfoot[L]{} % Empty left footer
\fancyfoot[C]{} % Empty center footer
\fancyfoot[R]{\thepage} % Page numbering for right footer
\renewcommand{\headrulewidth}{0pt} % Remove header underlines
\renewcommand{\footrulewidth}{0pt} % Remove footer underlines
\setlength{\headheight}{5.6pt} % Customize the height of the header

\numberwithin{equation}{section} % Number equations within sections (i.e. 1.1, 1.2, 2.1, 2.2 instead of 1, 2, 3, 4)
\numberwithin{figure}{section} % Number figures within sections (i.e. 1.1, 1.2, 2.1, 2.2 instead of 1, 2, 3, 4)
\numberwithin{table}{section} % Number tables within sections (i.e. 1.1, 1.2, 2.1, 2.2 instead of 1, 2, 3, 4)

\setlength\parindent{0pt} % Removes all indentation from paragraphs - comment this line for an assignment with lots of text

%----------------------------------------------------------------------------------------
%	TITLE SECTION
%----------------------------------------------------------------------------------------

\newcommand{\horrule}[1]{\rule{\linewidth}{#1}} % Create horizontal rule command with 1 argument of height

\title{	
\horrule{0.5pt} \\[0.2cm] % Thin top horizontal rule
\huge Units \\[-0.2cm] % The assignment title
\horrule{2pt} %\\[0.5cm] % Thick bottom horizontal rule
}

\author{Ian Ochs} % Your name

\date{\normalsize\today} % Today's date or a custom date

\begin{document}

\maketitle % Print the title

%----------------------------------------------------------------------------------------
%	PROBLEM 1
%----------------------------------------------------------------------------------------

\section*{Maxwell Equations}

In Yee algorithm, using Gaussian unit style with $c = 1$, i.e.
\begin{align}
	\nabla \times E &= -\frac{1}{c} \frac{\partial B}{\partial t} = -\frac{\partial B}{\partial \tilde{t}}\\
	\nabla \times B &= \frac{4 \pi}{c} J + \frac{1}{c} \frac{\partial E}{\partial t} = 4 \pi \tilde{J} + \frac{\partial E}{\partial \tilde{t}}.
\end{align}

I chose to keep cm as the length unit, so that the subsequent two Maxwell equations are unaffected:
\begin{align}
	\nabla \cdot E &= 4 \pi \rho \\
	\nabla \cdot B &= 0.
\end{align}

This means that the time has the unit of (1 cm) / (speed of light in seconds), i.e.
\begin{equation}
	1 \tilde{s} = \frac{1 \text{ cm}}{2.9979 * 10^{10} \text{ cm/s}} \approx 3.3 \times 10^{-11} \text{ s}.
\end{equation}

\section*{Velocity, acceleration, and current density}

The velocity $\tilde{v}$ is now normalized automatically to the speed of light, since it is measured in cm / (distance light travels in 1 cm).

We then have the Lorentz acceleration as
\begin{align}
	a &= \left(\frac{1}{2.9979 * 10^{10}}\right)^2 \frac{q}{m} \left( E + \tilde{v} \times B \right)\\
	&\approx 1.1\times 10^{-20} \frac{q}{m} \left( E + \tilde{v} \times B \right).
\end{align}
The small acceleration reflects the timescale separation between light waves and plasma motion.

Finally, current densities should be calculated in $\text{statC} / \text{cm}^2 \tilde{s}$.





%------------------------------------------------------------

\end{document}