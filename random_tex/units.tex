%%%%%%%%%%%%%%%%%%%%%%%%%%%%%%%%%%%%%%%%%
% Short Sectioned Assignment
% LaTeX Template
% Version 1.0 (5/5/12)
%
% This template has been downloaded from:
% http://www.LaTeXTemplates.com
%
% Original author:
% Frits Wenneker (http://www.howtotex.com)
%
% License:
% CC BY-NC-SA 3.0 (http://creativecommons.org/licenses/by-nc-sa/3.0/)
%
%%%%%%%%%%%%%%%%%%%%%%%%%%%%%%%%%%%%%%%%%

%----------------------------------------------------------------------------------------
%	PACKAGES AND OTHER DOCUMENT CONFIGURATIONS
%----------------------------------------------------------------------------------------

\documentclass[paper=a4, fontsize=11pt]{scrartcl} % A4 paper and 11pt font size

\usepackage[T1]{fontenc} % Use 8-bit encoding that has 256 glyphs
%\usepackage{fourier} % Use the Adobe Utopia font for the document - comment this line to return to the LaTeX default
\usepackage[english]{babel} % English language/hyphenation
\usepackage{amsmath,amsfonts,amsthm} % Math packages

\usepackage{lipsum} % Used for inserting dummy 'Lorem ipsum' text into the template

\usepackage{sectsty} % Allows customizing section commands
\allsectionsfont{\centering \normalfont\scshape} % Make all sections centered, the default font and small caps

\usepackage{fancyhdr} % Custom headers and footers
\pagestyle{fancyplain} % Makes all pages in the document conform to the custom headers and footers
\fancyhead{} % No page header - if you want one, create it in the same way as the footers below
\fancyfoot[L]{} % Empty left footer
\fancyfoot[C]{} % Empty center footer
\fancyfoot[R]{\thepage} % Page numbering for right footer
\renewcommand{\headrulewidth}{0pt} % Remove header underlines
\renewcommand{\footrulewidth}{0pt} % Remove footer underlines
\setlength{\headheight}{5.6pt} % Customize the height of the header

\numberwithin{equation}{section} % Number equations within sections (i.e. 1.1, 1.2, 2.1, 2.2 instead of 1, 2, 3, 4)
\numberwithin{figure}{section} % Number figures within sections (i.e. 1.1, 1.2, 2.1, 2.2 instead of 1, 2, 3, 4)
\numberwithin{table}{section} % Number tables within sections (i.e. 1.1, 1.2, 2.1, 2.2 instead of 1, 2, 3, 4)

\setlength\parindent{0pt} % Removes all indentation from paragraphs - comment this line for an assignment with lots of text

%----------------------------------------------------------------------------------------
%	TITLE SECTION
%----------------------------------------------------------------------------------------

\newcommand{\horrule}[1]{\rule{\linewidth}{#1}} % Create horizontal rule command with 1 argument of height

\title{	
\horrule{0.5pt} \\[0.2cm] % Thin top horizontal rule
\huge Units \\[-0.2cm] % The assignment title
\horrule{2pt} %\\[0.5cm] % Thick bottom horizontal rule
}

\author{EMOOPIC} % Your name

\date{\normalsize\today} % Today's date or a custom date

\begin{document}

\maketitle % Print the title

%----------------------------------------------------------------------------------------
%	PROBLEM 1
%----------------------------------------------------------------------------------------

\section*{Maxwell Equations}

In Yee algorithm, using Gaussian unit style with $c = 1$, i.e.
\begin{align}
	\nabla \times E &= -\frac{1}{c} \frac{\partial B}{\partial t} = -\frac{\partial B}{\partial \tilde{t}}\\
	\nabla \times B &= \frac{4 \pi}{c} J + \frac{1}{c} \frac{\partial E}{\partial t} = 4 \pi \tilde{J} + \frac{\partial E}{\partial \tilde{t}}.
\end{align}

We choose to keep cm as the length unit, so that the subsequent two Maxwell equations are unaffected:
\begin{align}
	\nabla \cdot E &= 4 \pi \rho \\
	\nabla \cdot B &= 0.
\end{align}

This means that the time has the unit of (1 cm) / (speed of light), i.e.
\begin{equation}
	1 \tilde{s} = \frac{1 \text{cm}}{c_0 \text{cm/s}} \approx 3.34 \times 10^{-11} \text{ s},
\end{equation}
where $c_0 \approx 2.9979\times 10^{10}$ is the speed of light in unit of cm/s.

\section*{Velocity and acceleration}

The velocity $\tilde{v}$ is now normalized automatically to the speed of light, since it is measured in cm / (distance light travels in 1 cm). Namely,
\begin{equation}
	\tilde{v}=\frac{v}{c_0 \text{cm/s}}.
\end{equation}
The non-relativistic thermal velocity $v_{th}$, defined as the mean of the magnitude of the velocity, is then
\begin{eqnarray}
	\tilde{v}_{th}&=&\frac{v_{th}}{c}=\sqrt{\frac{8T}{\pi mc^2}}\\
         &\approx&2.232\times 10^{-3}\sqrt{\frac{T/1 \text{eV}}{m/m_e}}.
\end{eqnarray}

We then have the Lorentz acceleration as
\begin{align}
	a &= \frac{1}{c_0^2} \frac{q}{m} \left( E + \tilde{v} \times B \right)\\
	&\approx 1.1\times 10^{-21} \frac{q}{m} \left( E + \tilde{v} \times B \right).
\end{align}
The small acceleration reflects the timescale separation between light waves and plasma motion.

\section*{Charge, Mass, and Field}
It is convenient to normalize charge $q$ to electron charge $e$, and normalize mass $m$ to electron mass $m_e$. Namely,
\begin{equation}
	\tilde{q}=\frac{q}{e},\hspace{3pt} \tilde{m}=\frac{m}{m_e}.
\end{equation}
Recall the value
\begin{equation}
	\frac{e}{m_e}\approx 5.2728\times10^{17} \text{statC}/\text{g}.
\end{equation}

Having normalized charge and mass, it is convenient to normalize the electric field $E$ to kilo statV/cm (KstatV/cm), and normalize the magnetic field $B$ to kilo Gauss (KG). Namely,
\begin{equation}
	\tilde{E}=\frac{E}{\text{KstatV/cm}},\hspace{5pt} \tilde{B}=\frac{B}{\text{KG}}.
\end{equation}
It is useful to note
\begin{equation}
	\frac{1\cdot\text{KstatV/cm}}{c_0 \text{cm/s}\cdot\text{KG}}=1.
\end{equation}
With such normalization, there is no large or small coefficient in the Newton's equation with Lorentz force. Explicitly, we have
\begin{eqnarray}
	\tilde{a}&=&\frac{a}{\text{cm}\cdot\tilde{s}^{-2}} \\
                 &=& \frac{\tilde{q}}{\tilde{m}}(\tilde{E}+\tilde{v}\times\tilde{B})
			\times\frac{e\cdot\text{KstatV/cm}}{m_e\cdot\text{cm}\cdot\tilde{s}^{-2}} \\
                 &\approx& 0.58668774 \frac{\tilde{q}}{\tilde{m}}(\tilde{E}+\tilde{v}\times\tilde{B}).
\end{eqnarray}
To get a sense of how large the electric and magnetic fields are, it is useful to note
\begin{eqnarray}
	\tilde{E}=1&\leftrightarrow& E\approx 299.79\text{KV/cm},\\
	\tilde{B}=1&\leftrightarrow& B=0.1\text{T},
\end{eqnarray}
where KV/cm stands for kilo volte/cm, and T stands for tesla. Notice that a constant field $\tilde{E}=1$ is a very large electric field, which can accelerate an electron to relativistic velocity within 1 $\tilde{s}$. In comparison, a constant field $\tilde{B}=1$ is moderate magnetic field that can be easily produced in laboratory.

\section*{Charge and Current Density}
Using the above units, the current density due to a single fluid species can be expressed as
\begin{eqnarray}
    \tilde{J}&=&\frac{J}{c}=\frac{qvn}{c}\\
             &=&\tilde{q}\tilde{v}\tilde{n} \times e \text{cm}^{-3} \\
             &=&4.8032\times 10^{-13}\frac{\text{KStatC}}{\text{cm}^{3}}
                \tilde{q}\tilde{v}\tilde{n}.
\end{eqnarray}
That is to say, current densities used in the program is in units $\text{KstatC}/\text{cm}^3$.

Similarly, the charge density can be expressed as
\begin{eqnarray}
    \rho&=&qn=\tilde{q}\tilde{n} \times e \text{cm}^{-3} \\
             &=&4.8032\times 10^{-13}\frac{\text{KStatC}}{\text{cm}^{3}}\tilde{q}\tilde{n}.
\end{eqnarray}
Notice that the charge density has the same unit as the current density.

\section*{Super Particles}
It is rare that one has the numerical resources to run a PIC code with physical number of particles. Therefore, super particles are typically used, each representing $N_s$ number of real particles. The mass and charge of a super particle is thereof $N_s$ time those of a real particle:
\begin{equation}
    m_s=N_s m, \hspace{3pt} e_s=N_s e.
\end{equation}
Since the charge-to-mass ratio remains the same, acceleration of a super particle is no different than that of a real particle. Hence, the velocity of super particle should not be scaled:
\begin{equation}
    v_s=v.
\end{equation}
Consequently, the temeprature of super particles should be $N_s$ times larger
\begin{equation}
    T_s\propto m_s v_{Ts}^2=N_s m v_T^2\propto N_s T.
\end{equation}
Nevertheless, since super particle density $n_s$, necessary for representing physical density $n_0$, is reduced to
\begin{equation}
    n_s=n_0/N_s,
\end{equation}
the plasma frequency and Debye length remains unchanged:
\begin{eqnarray}
  \omega_{ps}^2&=&\frac{4\pi n_se_s^2}{m_s}=\frac{4\pi(n_0/N_s)(N_s e)^2}{N_s m}=\omega_p^2,\\
  \lambda_{Ds}^2&=&\frac{T_s}{4\pi n_s e_s^2}=\frac{N_s T}{4\pi n_0/N_s (N_s e)^2}=\lambda_D^2.
\end{eqnarray}
As the super fraction $N_s=n_0/n_s$ becomes closer to $1$, the simulation becomes closer to the reality.
%------------------------------------------------------------

\end{document}
